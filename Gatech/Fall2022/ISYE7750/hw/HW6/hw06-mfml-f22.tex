\documentclass[11pt]{article}

\usepackage{amsmath,amssymb,graphicx,mathrsfs}
\usepackage[colorlinks=true,urlcolor=blue]{hyperref}
\usepackage{color}

% format and layout
\parindent = 0 pt
\parskip = 8 pt 
\addtolength{\textwidth}{1in}
\addtolength{\oddsidemargin}{-0.5in}
\addtolength{\textheight}{1.6in}
\addtolength{\topmargin}{-0.8in}

% time and date in footer
\usepackage{scrtime} 
\usepackage[draft,scrtime]{prelim2e}
\renewcommand{\PrelimWords}{Last updated \thistime,~\today}
\renewcommand{\PrelimText}{\textnormal{\tiny\PrelimTextStyle \PrelimWords}}

\graphicspath{{./Figs/}}

%--------------------------------------------------------------------

% fields
\newcommand{\R}{\mathbb{R}}
\newcommand{\C}{\mathbb{C}}
\newcommand{\Z}{\mathbb{Z}}
\newcommand{\N}{\mathbb{N}}

% complex numbers
\renewcommand{\Re}[1]{\operatorname{Re}\left\{#1\right\}}
\renewcommand{\Im}[1]{\operatorname{Im}\left\{#1\right\}}
\newcommand{\conj}[1]{\mkern 1.5mu\overline{\mkern-1.5mu#1\mkern-1.5mu}\mkern 1.5mu}

% probability and stat
\renewcommand{\P}[1]{\operatorname{P}\left(#1\right)}
\newcommand{\E}{\operatorname{E}}
\newcommand{\var}{\operatorname{var}}

% calculus 
\renewcommand{\d}[1]{d#1}

% constants (written in roman, if wanted)
\newcommand{\e}{e}
\renewcommand{\j}{j}

% linear algebra
% 	vector notation
\newcommand{\vct}[1]{\boldsymbol{#1}}
%   matrices
\newcommand{\mtx}[1]{\boldsymbol{#1}}
%   block vector
\newcommand{\bvct}[1]{\mathbf{#1}}
%   block matrix
\newcommand{\bmtx}[1]{\mathbf{#1}}
%	inner products
\newcommand{\<}{\langle}
\renewcommand{\>}{\rangle}
% 	transpose, Hermitian, pseudo-inverse
\renewcommand{\H}{\mathrm{H}}
\newcommand{\T}{\mathrm{T}}
\newcommand{\pinv}{\dagger}
%	fundamental subspaces
\newcommand{\Null}{\operatorname{Null}}
\newcommand{\Range}{\operatorname{Range}}
\newcommand{\Span}{\operatorname{Span}}
\newcommand{\Row}{\operatorname{Row}}
\newcommand{\Column}{\operatorname{Col}}
%	operators
\newcommand{\trace}{\operatorname{trace}}
\newcommand{\rank}{\operatorname{rank}}

% sets and topology
\newcommand{\set}[1]{\mathcal{#1}}
\newcommand{\closure}{\operatorname{cl}}	% closure
\newcommand{\interior}{\operatorname{int}}
\newcommand{\boundary}{\operatorname{bd}}
\newcommand{\diameter}{\operatorname{diam}}

% functional analysis
\newcommand{\domain}{\operatorname{dom}}
\newcommand{\epigraph}{\operatorname{epi}}
\newcommand{\hypograph}{\operatorname{hypo}}
%\newcommand{\linop}[1]{\mathcal{#1}}	% general linear operator
\newcommand{\linop}[1]{\mathscr{#1}}	% general linear operator

% optimization
\DeclareMathOperator*{\minimize}{\text{minimize}}
\DeclareMathOperator*{\maximize}{\text{maximize}}
\DeclareMathOperator*{\argmin}{\text{arg~min}}
\DeclareMathOperator*{\argmax}{\text{arg~max}}


%--------------------------------------------------------------------------

\newcommand{\va}{\vct{a}}
\newcommand{\vb}{\vct{b}}
\newcommand{\vc}{\vct{c}}
\newcommand{\vd}{\vct{d}}
\newcommand{\ve}{\vct{e}}
\newcommand{\vf}{\vct{f}}
\newcommand{\vg}{\vct{g}}
\newcommand{\vh}{\vct{h}}
\newcommand{\vi}{\vct{i}}
\newcommand{\vj}{\vct{j}}
\newcommand{\vk}{\vct{k}}
\newcommand{\vl}{\vct{l}}
\newcommand{\vm}{\vct{m}}
\newcommand{\vn}{\vct{n}}
\newcommand{\vo}{\vct{o}}
\newcommand{\vp}{\vct{p}}
\newcommand{\vq}{\vct{q}}
\newcommand{\vr}{\vct{r}}
\newcommand{\vs}{\vct{s}}
\newcommand{\vt}{\vct{t}}
\newcommand{\vu}{\vct{u}}
\newcommand{\vv}{\vct{v}}
\newcommand{\vw}{\vct{w}}
\newcommand{\vx}{\vct{x}}
\newcommand{\vy}{\vct{y}}
\newcommand{\vz}{\vct{z}}
%
\newcommand{\valpha}{\vct{\alpha}}
\newcommand{\vbeta}{\vct{\beta}}
\newcommand{\vgamma}{\vct{\gamma}}
\newcommand{\vdelta}{\vct{\delta}}
\newcommand{\vepsilon}{\vct{\epsilon}}
\newcommand{\vzeta}{\vct{\zeta}}
\newcommand{\veta}{\vct{\eta}}
\newcommand{\vtheta}{\vct{\theta}}
\newcommand{\viota}{\vct{\iota}}
\newcommand{\vkappa}{\vct{\kappa}}
\newcommand{\vlambda}{\vct{\lambda}}
\newcommand{\vmu}{\vct{\mu}}
\newcommand{\vnu}{\vct{\nu}}
\newcommand{\vxi}{\vct{\xi}}
\newcommand{\vomicron}{\vct{\omicron}}
\newcommand{\vpi}{\vct{\pi}}
\newcommand{\vrho}{\vct{\rho}}
\newcommand{\vsigma}{\vct{\sigma}}
\newcommand{\vtau}{\vct{\tau}}
\newcommand{\vupsilon}{\vct{\upsilon}}
\newcommand{\vphi}{\vct{\phi}}
\newcommand{\vchi}{\vct{\chi}}
\newcommand{\vpsi}{\vct{\psi}}
\newcommand{\vomega}{\vct{\omega}}
%
\newcommand{\vzero}{\vct{0}}
\newcommand{\vone}{\vct{1}}

%------------------------------------------------

\newcommand{\mA}{\mtx{A}}
\newcommand{\mB}{\mtx{B}}
\newcommand{\mC}{\mtx{C}}
\newcommand{\mD}{\mtx{D}}
\newcommand{\mE}{\mtx{E}}
\newcommand{\mF}{\mtx{F}}
\newcommand{\mG}{\mtx{G}}
\newcommand{\mH}{\mtx{H}}
\newcommand{\mJ}{\mtx{J}}
\newcommand{\mK}{\mtx{K}}
\newcommand{\mL}{\mtx{L}}
\newcommand{\mM}{\mtx{M}}
\newcommand{\mN}{\mtx{N}}
\newcommand{\mO}{\mtx{O}}
\newcommand{\mP}{\mtx{P}}
\newcommand{\mQ}{\mtx{Q}}
\newcommand{\mR}{\mtx{R}}
\newcommand{\mS}{\mtx{S}}
\newcommand{\mT}{\mtx{T}}
\newcommand{\mU}{\mtx{U}}
\newcommand{\mV}{\mtx{V}}
\newcommand{\mW}{\mtx{W}}
\newcommand{\mX}{\mtx{X}}
\newcommand{\mY}{\mtx{Y}}
\newcommand{\mZ}{\mtx{Z}}
%
\newcommand{\mGamma}{\mtx{\Gamma}}
\newcommand{\mDelta}{\mtx{\Delta}}
\newcommand{\mTheta}{\mtx{\Theta}}
\newcommand{\mLambda}{\mtx{\Lambda}}
\newcommand{\mXi}{\mtx{\Xi}}
\newcommand{\mPi}{\mtx{\Pi}}
\newcommand{\mSigma}{\mtx{\Sigma}}
\newcommand{\mUpsilon}{\mtx{\Upsilon}}
\newcommand{\mPhi}{\mtx{\Phi}}
\newcommand{\mPsi}{\mtx{\Psi}}
\newcommand{\mOmega}{\mtx{\Omega}}
%
\newcommand{\mzero}{\mtx{0}}
\newcommand{\mone}{\mtx{1}}
\newcommand{\mId}{\mathbf{I}}
\newcommand{\mEx}{\mathbf{J}}


\newcommand{\mAbar}{\underline{\mtx{A}}}
\newcommand{\mRbar}{\underline{\mtx{R}}}
\newcommand{\vebar}{\underline{\vct{e}}}
\newcommand{\vxbar}{\underline{\vct{x}}}
\newcommand{\vybar}{\underline{\vct{y}}}

%------------------------------------------------
\newcommand{\loA}{\linop{A}}
\newcommand{\loB}{\linop{B}}
\newcommand{\loC}{\linop{C}}
\newcommand{\loD}{\linop{D}}
\newcommand{\loE}{\linop{E}}
\newcommand{\loF}{\linop{F}}
\newcommand{\loG}{\linop{G}}
\newcommand{\loH}{\linop{H}}
\newcommand{\loI}{\linop{I}}
\newcommand{\loJ}{\linop{J}}
\newcommand{\loK}{\linop{K}}
\newcommand{\loL}{\linop{L}}
\newcommand{\loM}{\linop{M}}
\newcommand{\loN}{\linop{N}}
\newcommand{\loO}{\linop{O}}
\newcommand{\loP}{\linop{P}}
\newcommand{\loQ}{\linop{Q}}
\newcommand{\loR}{\linop{R}}
\newcommand{\loS}{\linop{S}}
\newcommand{\loT}{\linop{T}}
\newcommand{\loU}{\linop{U}}
\newcommand{\loV}{\linop{V}}
\newcommand{\loW}{\linop{W}}
\newcommand{\loX}{\linop{X}}
\newcommand{\loY}{\linop{Y}}
\newcommand{\loZ}{\linop{Z}}

%------------------------------------------------

\newcommand{\setA}{\set{A}}
\newcommand{\setB}{\set{B}}
\newcommand{\setC}{\set{C}}
\newcommand{\setD}{\set{D}}
\newcommand{\setE}{\set{E}}
\newcommand{\setF}{\set{F}}
\newcommand{\setG}{\set{G}}
\newcommand{\setH}{\set{H}}
\newcommand{\setI}{\set{I}}
\newcommand{\setJ}{\set{J}}
\newcommand{\setK}{\set{K}}
\newcommand{\setL}{\set{L}}
\newcommand{\setM}{\set{M}}
\newcommand{\setN}{\set{N}}
\newcommand{\setO}{\set{O}}
\newcommand{\setP}{\set{P}}
\newcommand{\setQ}{\set{Q}}
\newcommand{\setR}{\set{R}}
\newcommand{\setS}{\set{S}}
\newcommand{\setT}{\set{T}}
\newcommand{\setU}{\set{U}}
\newcommand{\setV}{\set{V}}
\newcommand{\setW}{\set{W}}
\newcommand{\setX}{\set{X}}
\newcommand{\setY}{\set{Y}}
\newcommand{\setZ}{\set{Z}}



%--------------------------------------------------------------------

\begin{document}

%--------------------------------------------------------------------

\noindent
{\bf\large Math Foundations of ML, Fall 2022} \\[3mm]
{\bf\large Homework \#6} \\[3mm]
{\bf Due Monday November 14, at 5:00pm ET}\\[3mm]
%{\bf Reading: XXXX}

{\bf As stated in the syllabus, unauthorized use of previous semester course materials is strictly prohibited in this course.}

\begin{enumerate}
%--------------------------------------------------------------------


\item Suppose that two random variables $(X,Y)$ have joint pdf $f_{X,Y}(x,y)$.  Find an expression for the pdf $f_Z(z)$ where $Z=X+Y$.  You can start by realizing that 
\[
	F_Z(u|X=\beta) = \P{Z\leq u|X=\beta} = \P{Y\leq u-\beta|X=\beta}.
\]
You can combine the expressions above by integrating over $\beta$, and see that the resulting expression corresponds to an integral of $f_{X,Y}(x,y)$ over a half plane.  From this, you can get the pdf for $Z$ by applying the Fundamental Theorem of Calculus.  How does your expression simplify if $X$ and $Y$ are independent?  (Convolution!)

%--- 
\vspace{4mm} 
\item Let $X_1,X_2,\ldots$ be independent uniform random variables,
\[
	X_n\sim\mathrm{Uniform}(-1/2,1/2),\quad\text{meaning}\quad 
	f_X(x) = \begin{cases} 1, & -1/2\leq x\leq 1/2 \\ 0, & \text{otherwise.} \end{cases}
\]
\begin{enumerate}

	\item What is the density function for $Y=X_1+X_2+X_3$?  (If you compute this correctly, you will meet an old friend.)
	
	\item The \emph{moment generating function} of a random variable is
	\[
		\varphi_X(t) = \E\left[\e^{tX}\right].
	\]
	It is a fact that if $\varphi_X(t)=\varphi_W(t)$ for all $t$, then $X$ and $W$ have the same distribution.  It is a fact that if $G\sim\mathrm{Normal}(0,\sigma^2)$, then $\varphi_G(t) = \e^{\sigma^2t^2/2}$.  Let
	\[
		Y_N = \frac{1}{\sqrt{N}}\sum_{n=1}^NX_n.
	\]
	Find an expression for $\varphi_{Y_N}(t)$.  Plot $\varphi_{Y_N}(t)$ and $\varphi_G(t)$ for $\sigma^2=\var(Y)=\var(X_n) = 1/12$ on the same set of axes for $N=1,2,5,10$ and $0\leq t\leq 5$.  What might you conclude about $Y_N$ as $N\rightarrow\infty$? (\emph{\textbf{Bonus question}: argue rigorously that $\varphi_{Y_N}(t)\rightarrow\varphi_G(t)$ for all $t$.})
	
	\item It is a fact that if $\phi(z)$ is a monotonically increasing function, then for any random variable $Z$,
	\[
		\P{Z > u} = \P{\phi(Z) >\phi(u)}. 
	\]
	Use $\phi(z) = e^{tz}$ and the Markov inequality to derive a bound on $\P{Z_N > u}$, where
	\[
		Z_N = \frac{1}{N}\sum_{n=1}^NX_n.
	\]
	For the special case of $t=4u/N$, compare this bound, as a function of $u$, to that obtained using the Chebsyshev inequality.
\end{enumerate}


%--------------------------------------------------------------------
\vspace{4mm}
\item Let $Z_1,\ldots,Z_N$ be a sequence of independent Gaussian random variables with mean $0$ and variance $1$.  You observe the random vector $X$ in $\R^N$ that is generated through the autoregressive process
\[
	X_k = \begin{cases} Z_1, &k=1 \\ aX_{k-1}+Z_k, &k > 1.\end{cases}
\]	
Given $X=\vx$, find the MLE for $a\in\R$.  (Hint: Conditional independence.)  (Further hint: The conditional independence structure makes this a Markov process, meaning that we can factor the distribution for $X\in\R^N$ as 
\[
	f_X(\vx) = f_{X_1}(x_1)f_{X_2}(x_2|x_1)f_{X_3}(x_3|x_2)\cdots f_{X_N}(x_N|x_{N-1}).
\]
)

%--------------------------------------------------------------------
\vspace{4mm} 
\item Let $X$ be a Gaussian random vector taking values in $\R^N$, let $E$ be a Gaussian random vector taking values in $\R^M$, and let $\mA$ be a $M\times N$ matrix.  We have
\[
	X\sim\mathrm{Normal}(\vzero,\mR_x),\quad
	E\sim\mathrm{Normal}(\vzero,\mR_e),\quad
	\text{$X,E$ independent}.
\]
We will make observation of the random vector
\[
	Y = \mA X + E.
\]
\begin{enumerate}

	\item From the lecture notes, it is clear that $Y$ is a Gaussian random vector in $\R^M$ and that $\E[Y]=\vzero$.  Find the covariance matrix for the Gaussian random vector $\begin{bmatrix} X \\ Y \end{bmatrix}$ that takes values in $\R^{N+M}$.
	
	\item Suppose we observe $Y=\vy$.  What is the minimum mean-square error estimate of $X$ given $Y=\vy$?
	
	\item Suppose $\mR_x = \sigma_x^2\mId$ and $\mR_e = \sigma_e^2\mId$.  In this case, your MMSE estimator should look familiar, and you should see immediately that $\hat{x}_{MMSE}$ is in the row space of $\mA$.  What are the $\hat{\alpha}_n$ is the expression below?
	\[
		\hat{\vx}_{MMSE} = \sum_{n=1}^N\alpha_n\vv_n,\quad\text{where the $\vv_n$ are the right singular vectors of $\mA$}.
	\]
	
	\item Take $\mR_x$ and $\mR_e$ as in part (c), and assume that $\mA$ has full column rank.  What is MSE $\E[\|\hat{\vx}_{MMSE}-X\|_2^2]$ of the MMSE estimate $\hat{\vx}_{MMSE}$?
\end{enumerate}


%--- 
\vspace{4mm} 
\item Let $\mA$ be an $M\times N$ matrix with full column rank.  Let $E$ be a Gaussian random vector in $\R^M$ with mean $\vzero$ and covariance $\mR_e$.   Suppose we observe
\[
	Y = \mA\vtheta_0 + E,
\]
where $\vtheta\in\R^N$ is unknown.
\begin{enumerate}
	\item What is the distribution of $Y$ and how does it depend on $\vtheta_0$?
	\item Find a closed form expression for the maximum likelihood estimate of $\vtheta_0$.  (In this case, we are working from a single sample of a random vector.)
	\item What is the distribution of the MLE estimator $\hat\mTheta$?  Is $\hat\mTheta$ unbiased?
	\item What is the MSE of the MLE, $\E[\|\hat\mTheta-\vtheta_0\|_2^2]$?
	\item Compute the Fisher information matrix $\mJ(\vtheta_0)$ and verify that the MLE meets the Cramer-Rao lower bound.
	\item Defend the following statement: The MLE is the best unbiased estimator of $\vtheta_0$.  
\end{enumerate}

%--- 
\vspace{4mm} 
\item A Cauchy random variable with ``location parameter'' $\nu$ has a density function
\begin{equation}
	\label{eq:cauchyrv}
	f_X(x;\nu) = \frac{1}{\pi(1+(x-\nu)^2)},\quad x\in\R.
\end{equation}
Despite its simple definition, this is a strange animal.  First of all, its mean is not defined, as the integral $\int x/(1+x^2)~\d{x}$ is not absolutely convergent.  It is also easy to see that the variance is infinite.  But as you can see (especially if you sketch it), the density is symmetric around $\nu$, and $\nu$ is certainly the median.  

 Let $X_1,X_2,\ldots,X_N$ be iid Cauchy random variables distributed as in \eqref{eq:cauchyrv}.  From observed data $X_1=x_1,\ldots,X_N=x_N$, we will compare three estimators: the sample mean
	\[
		\hat\nu_{mn} = \frac{1}{N}\sum_{n=1}^Nx_n,
	\]
	the sample median 
	\[
		\hat\nu_{md} = \begin{cases}
			x_{((N+1)/2)}, & $N$\text{ odd}, \\[2mm]
			\frac{x_{(N/2)}+x_{(N/2+1)}}{2}, & $N$\text{ even},
		\end{cases}
	\]
	where $x_{(i)}$ is the $i$th largest value in $\{x_1,\ldots,x_N\}$, and the MLE
	\[
		\hat\nu_{mle} = \argmax_\nu L(\nu;x_1,\ldots,x_N) = \argmax_{\nu}\sum_{n=1}^N\ell(\nu;x_n)
	\]
	where $\ell(\nu;x_n) = \log f_X(x_n;\nu)$.
\begin{enumerate}
	\item One particular draw of data for $N=50$ is variable \texttt{x} in the file \texttt{hw06p6a.mat}.  Plot the log likelihood function, and report the MLE for $\nu$.  Your MLE will of course be approximate, but make sure yours is accurate to within $10^{-2}$ to the true MLE.  I will give you a hint here and tell you that the true value of $\nu$ is somewhere in the interval $[0,5]$.
	
	\item The file \texttt{hw06p6b.mat} contains a matrix \texttt{X}.  This is an $N\times Q$ matrix, where $N=50$ and $Q=1000$; each entry is an independent Cauchy random variable with $\nu_0=3$.  Treating each column of \texttt{X} as a single draw of the data for $N=50$, compute the sample mean, sample median, and MLE for each column.  From these, report the empirical mean squared error (by averaging $(\hat\nu-\nu_0)^2$ over all $Q$ trials) for each of the three estimators.
	
	\item Find an integral expression for the expected log likelihood function $e(\nu) = \E[\ell(\nu;X)]$ when $X$ has Cauchy density $f_X(x;\nu_0)$ as in \eqref{eq:cauchyrv}.  Your expression should have the form
	\[
		e(\nu) = \int_{-\infty}^\infty \text{(something that depends on $x,\nu,\nu_0$)}~\d{x}.
	\]
	Compute $e(\nu)$ for $\nu_0=3$ for 250 equally spaced values of $\nu$ between $0$ and $5$.  You can do this using numerical integration (the \texttt{integral} function in MATLAB or \texttt{scipy.integrate.quad} in Python).  Make a plot of $e(\nu) = \E[\ell(\nu;X)]$.
	
	\item Plot, overlayed on the same axes, the (renormalized) log likelihood functions $\frac{1}{N}\sum_{n=1}^{N} \ell(\nu;x_n)$ as a function of $\nu\in[0,5]$ for each of the first $10$ columns of \texttt{X} from part (b).  On top of this, plot $e(\nu) = \E[\ell(\nu;X)]$ from part (c) as a dotted line.
	
\end{enumerate}

%--------------------------------------------------------------------
\end{enumerate}
\end{document}
