\documentclass[11pt]{article}

\usepackage{amsmath,amssymb,graphicx}
\usepackage[colorlinks=true,urlcolor=blue]{hyperref}
\usepackage{color}

% format and layout
\parindent = 0 pt
\parskip = 8 pt 
\addtolength{\textwidth}{1in}
\addtolength{\oddsidemargin}{-0.5in}
\addtolength{\textheight}{1.6in}
\addtolength{\topmargin}{-0.8in}

% time and date in footer
\usepackage{scrtime} 
\usepackage[draft,scrtime]{prelim2e}
\renewcommand{\PrelimWords}{Last updated \thistime,~\today}
\renewcommand{\PrelimText}{\textnormal{\tiny\PrelimTextStyle \PrelimWords}}

\graphicspath{{./Figs/}}

%--------------------------------------------------------------------

% fields
\newcommand{\R}{\mathbb{R}}
\newcommand{\C}{\mathbb{C}}
\newcommand{\Z}{\mathbb{Z}}
\newcommand{\N}{\mathbb{N}}

% complex numbers
\renewcommand{\Re}[1]{\operatorname{Re}\left\{#1\right\}}
\renewcommand{\Im}[1]{\operatorname{Im}\left\{#1\right\}}
\newcommand{\conj}[1]{\mkern 1.5mu\overline{\mkern-1.5mu#1\mkern-1.5mu}\mkern 1.5mu}

% probability and stat
\renewcommand{\P}[1]{\operatorname{P}\left(#1\right)}
\newcommand{\E}{\operatorname{E}}
\newcommand{\var}{\operatorname{var}}

% calculus 
\renewcommand{\d}[1]{d#1}

% constants (written in roman, if wanted)
\newcommand{\e}{e}
\renewcommand{\j}{j}

% linear algebra
% 	vector notation
\newcommand{\vct}[1]{\boldsymbol{#1}}
%   matrices
\newcommand{\mtx}[1]{\boldsymbol{#1}}
%   block vector
\newcommand{\bvct}[1]{\mathbf{#1}}
%   block matrix
\newcommand{\bmtx}[1]{\mathbf{#1}}
%	inner products
\newcommand{\<}{\langle}
\renewcommand{\>}{\rangle}
% 	transpose, Hermitian, pseudo-inverse
\renewcommand{\H}{\mathrm{H}}
\newcommand{\T}{\mathrm{T}}
\newcommand{\pinv}{\dagger}
%	fundamental subspaces
\newcommand{\Null}{\operatorname{Null}}
\newcommand{\Range}{\operatorname{Range}}
\newcommand{\Span}{\operatorname{Span}}
\newcommand{\Row}{\operatorname{Row}}
\newcommand{\Column}{\operatorname{Col}}
%	operators
\newcommand{\trace}{\operatorname{trace}}
\newcommand{\rank}{\operatorname{rank}}

% sets and topology
\newcommand{\set}[1]{\mathcal{#1}}
\newcommand{\closure}{\operatorname{cl}}	% closure
\newcommand{\interior}{\operatorname{int}}
\newcommand{\boundary}{\operatorname{bd}}
\newcommand{\diameter}{\operatorname{diam}}

% functional analysis
\newcommand{\domain}{\operatorname{dom}}
\newcommand{\epigraph}{\operatorname{epi}}
\newcommand{\hypograph}{\operatorname{hypo}}
%\newcommand{\linop}[1]{\mathcal{#1}}	% general linear operator
\newcommand{\linop}[1]{\mathscr{#1}}	% general linear operator

% optimization
\DeclareMathOperator*{\minimize}{\text{minimize}}
\DeclareMathOperator*{\maximize}{\text{maximize}}
\DeclareMathOperator*{\argmin}{\text{arg~min}}
\DeclareMathOperator*{\argmax}{\text{arg~max}}


%--------------------------------------------------------------------------

\newcommand{\va}{\vct{a}}
\newcommand{\vb}{\vct{b}}
\newcommand{\vc}{\vct{c}}
\newcommand{\vd}{\vct{d}}
\newcommand{\ve}{\vct{e}}
\newcommand{\vf}{\vct{f}}
\newcommand{\vg}{\vct{g}}
\newcommand{\vh}{\vct{h}}
\newcommand{\vi}{\vct{i}}
\newcommand{\vj}{\vct{j}}
\newcommand{\vk}{\vct{k}}
\newcommand{\vl}{\vct{l}}
\newcommand{\vm}{\vct{m}}
\newcommand{\vn}{\vct{n}}
\newcommand{\vo}{\vct{o}}
\newcommand{\vp}{\vct{p}}
\newcommand{\vq}{\vct{q}}
\newcommand{\vr}{\vct{r}}
\newcommand{\vs}{\vct{s}}
\newcommand{\vt}{\vct{t}}
\newcommand{\vu}{\vct{u}}
\newcommand{\vv}{\vct{v}}
\newcommand{\vw}{\vct{w}}
\newcommand{\vx}{\vct{x}}
\newcommand{\vy}{\vct{y}}
\newcommand{\vz}{\vct{z}}
%
\newcommand{\valpha}{\vct{\alpha}}
\newcommand{\vbeta}{\vct{\beta}}
\newcommand{\vgamma}{\vct{\gamma}}
\newcommand{\vdelta}{\vct{\delta}}
\newcommand{\vepsilon}{\vct{\epsilon}}
\newcommand{\vzeta}{\vct{\zeta}}
\newcommand{\veta}{\vct{\eta}}
\newcommand{\vtheta}{\vct{\theta}}
\newcommand{\viota}{\vct{\iota}}
\newcommand{\vkappa}{\vct{\kappa}}
\newcommand{\vlambda}{\vct{\lambda}}
\newcommand{\vmu}{\vct{\mu}}
\newcommand{\vnu}{\vct{\nu}}
\newcommand{\vxi}{\vct{\xi}}
\newcommand{\vomicron}{\vct{\omicron}}
\newcommand{\vpi}{\vct{\pi}}
\newcommand{\vrho}{\vct{\rho}}
\newcommand{\vsigma}{\vct{\sigma}}
\newcommand{\vtau}{\vct{\tau}}
\newcommand{\vupsilon}{\vct{\upsilon}}
\newcommand{\vphi}{\vct{\phi}}
\newcommand{\vchi}{\vct{\chi}}
\newcommand{\vpsi}{\vct{\psi}}
\newcommand{\vomega}{\vct{\omega}}
%
\newcommand{\vzero}{\vct{0}}
\newcommand{\vone}{\vct{1}}

%------------------------------------------------

\newcommand{\mA}{\mtx{A}}
\newcommand{\mB}{\mtx{B}}
\newcommand{\mC}{\mtx{C}}
\newcommand{\mD}{\mtx{D}}
\newcommand{\mE}{\mtx{E}}
\newcommand{\mF}{\mtx{F}}
\newcommand{\mG}{\mtx{G}}
\newcommand{\mH}{\mtx{H}}
\newcommand{\mJ}{\mtx{J}}
\newcommand{\mK}{\mtx{K}}
\newcommand{\mL}{\mtx{L}}
\newcommand{\mM}{\mtx{M}}
\newcommand{\mN}{\mtx{N}}
\newcommand{\mO}{\mtx{O}}
\newcommand{\mP}{\mtx{P}}
\newcommand{\mQ}{\mtx{Q}}
\newcommand{\mR}{\mtx{R}}
\newcommand{\mS}{\mtx{S}}
\newcommand{\mT}{\mtx{T}}
\newcommand{\mU}{\mtx{U}}
\newcommand{\mV}{\mtx{V}}
\newcommand{\mW}{\mtx{W}}
\newcommand{\mX}{\mtx{X}}
\newcommand{\mY}{\mtx{Y}}
\newcommand{\mZ}{\mtx{Z}}
%
\newcommand{\mGamma}{\mtx{\Gamma}}
\newcommand{\mDelta}{\mtx{\Delta}}
\newcommand{\mTheta}{\mtx{\Theta}}
\newcommand{\mLambda}{\mtx{\Lambda}}
\newcommand{\mXi}{\mtx{\Xi}}
\newcommand{\mPi}{\mtx{\Pi}}
\newcommand{\mSigma}{\mtx{\Sigma}}
\newcommand{\mUpsilon}{\mtx{\Upsilon}}
\newcommand{\mPhi}{\mtx{\Phi}}
\newcommand{\mPsi}{\mtx{\Psi}}
\newcommand{\mOmega}{\mtx{\Omega}}
%
\newcommand{\mzero}{\mtx{0}}
\newcommand{\mone}{\mtx{1}}
\newcommand{\mId}{\mathbf{I}}
\newcommand{\mEx}{\mathbf{J}}


\newcommand{\mAbar}{\underline{\mtx{A}}}
\newcommand{\mRbar}{\underline{\mtx{R}}}
\newcommand{\vebar}{\underline{\vct{e}}}
\newcommand{\vxbar}{\underline{\vct{x}}}
\newcommand{\vybar}{\underline{\vct{y}}}

%------------------------------------------------
\newcommand{\loA}{\linop{A}}
\newcommand{\loB}{\linop{B}}
\newcommand{\loC}{\linop{C}}
\newcommand{\loD}{\linop{D}}
\newcommand{\loE}{\linop{E}}
\newcommand{\loF}{\linop{F}}
\newcommand{\loG}{\linop{G}}
\newcommand{\loH}{\linop{H}}
\newcommand{\loI}{\linop{I}}
\newcommand{\loJ}{\linop{J}}
\newcommand{\loK}{\linop{K}}
\newcommand{\loL}{\linop{L}}
\newcommand{\loM}{\linop{M}}
\newcommand{\loN}{\linop{N}}
\newcommand{\loO}{\linop{O}}
\newcommand{\loP}{\linop{P}}
\newcommand{\loQ}{\linop{Q}}
\newcommand{\loR}{\linop{R}}
\newcommand{\loS}{\linop{S}}
\newcommand{\loT}{\linop{T}}
\newcommand{\loU}{\linop{U}}
\newcommand{\loV}{\linop{V}}
\newcommand{\loW}{\linop{W}}
\newcommand{\loX}{\linop{X}}
\newcommand{\loY}{\linop{Y}}
\newcommand{\loZ}{\linop{Z}}

%------------------------------------------------

\newcommand{\setA}{\set{A}}
\newcommand{\setB}{\set{B}}
\newcommand{\setC}{\set{C}}
\newcommand{\setD}{\set{D}}
\newcommand{\setE}{\set{E}}
\newcommand{\setF}{\set{F}}
\newcommand{\setG}{\set{G}}
\newcommand{\setH}{\set{H}}
\newcommand{\setI}{\set{I}}
\newcommand{\setJ}{\set{J}}
\newcommand{\setK}{\set{K}}
\newcommand{\setL}{\set{L}}
\newcommand{\setM}{\set{M}}
\newcommand{\setN}{\set{N}}
\newcommand{\setO}{\set{O}}
\newcommand{\setP}{\set{P}}
\newcommand{\setQ}{\set{Q}}
\newcommand{\setR}{\set{R}}
\newcommand{\setS}{\set{S}}
\newcommand{\setT}{\set{T}}
\newcommand{\setU}{\set{U}}
\newcommand{\setV}{\set{V}}
\newcommand{\setW}{\set{W}}
\newcommand{\setX}{\set{X}}
\newcommand{\setY}{\set{Y}}
\newcommand{\setZ}{\set{Z}}



%--------------------------------------------------------------------

\begin{document}

%--------------------------------------------------------------------

\noindent
{\bf\large Math Foundations of ML, Fall 2022} \hspace{2cm} {\bf\large Instructor: Justin Romberg} \\[3mm]
{\bf Homework 3} \\[3mm]
{\bf Due: Friday, October 7, 2022 at 5:00 pm EST}\\[3mm]
%{\bf Reading: XXXX}

{\bf As stated in the syllabus, unauthorized use of previous semester course materials is strictly prohibited in this course.}

\begin{enumerate}













%--------------------------------------------------------------------
\vspace{4mm} 
\item The file \texttt{hw3p1\_data.mat} contains two variables: \texttt{udata} and \texttt{ydata}.  We will use this data to estimate a function $f:\R^2\rightarrow\R$.  The columns of \texttt{udata} contain sample locations, of which there are $M=100$.  The entries of $\vy$ are the corresponding responses.  We want to estimate $f$ such that
\[
	f(\vu_m)\approx y_m,\quad m = 1,\ldots,M, \quad\text{where}\quad \vu_m = \begin{bmatrix} s_m \\ t_m \end{bmatrix}.
\]
We will restrict $f$ to be a second-order polynomial on $[0,1]\times[0,1]$:
\begin{equation}
	\label{eq:secondorderpoly}
	f(s,t) = \alpha_1s^2 + \alpha_2t^2 + \alpha_3st + \alpha_4s + \alpha_5t + \alpha_6,
\end{equation}
which means that $f$ lies in a six dimensional subspace of $L_2([0,1]^2)$.
\begin{enumerate}
	\item Explain how to compute the $100\times 6$ matrix $\mA$ so that $\vy\approx\mA\valpha$, where $\vy$ contains the $100$ response values in \texttt{ydata}.  Write the code to compute $\mA$ and turn it in.
	\item Solve 
	\[
		\minimize_{\valpha\in\R^6}~\|\vy-\mA\valpha\|_2^2.
	\] 
	Turn in your code and the numerical value of your solution $\hat\valpha\in\R^6$.
	\item Make a contour plot of the corresponding 
	\[
		\hat{f}(s,t) =  \hat\alpha_1s^2 + \hat\alpha_2t^2 + \hat\alpha_3st + \hat\alpha_4s + \hat\alpha_5t + \hat\alpha_6.
	\]
	Include 50 contour lines, just so we have a very clear picture of what this function looks like.
\end{enumerate}












%--------------------------------------------------------------------
\vspace{4mm} 
\item Consider the space $\setP_2$ of  second-order polynomials on $[0,1]^2$ specified by $\valpha\in\R^6$ as in \eqref{eq:secondorderpoly} above.
\begin{enumerate}

	\item At every point $(s,t)$, the gradient $\nabla f(s,t)$ of a function $f \in \setP_2$  is a vector in $\R^2$.  As every $f \in \setP_2$ is specified by a vector $\valpha\in\R^6$, we can think of the gradient at $(s,t)$ as a mapping from $\R^6$ to $\R^2$.  Show that this mapping is linear, which means, for a specified $(s,t)$, there is a $2\times 6$ matrix $\mG_{s,t} \in \mathbb{R}^{2 \times 6}$ such that
	\[
		\nabla f(s,t) = \mG_{s,t}\valpha
	\]
	
	\item Find the $6\times 6$ matrix $\mH_{s,t} \in \mathbb{R}^{6 \times 6}$ such that\footnote{Hint: $\|\mG_{s,t}\valpha\|_2^2 = \<\mG_{s,t}\valpha,\mG_{s,t}\valpha\> = \cdots$.}
	\[
		\|\nabla f(s,t)\|_2^2 = \valpha^\T \mH_{s,t}\valpha.
	\]
	What kinds of functions $f$ are in the null space of $\mH_{s,t}$ for all $s$ and $t$?  Why?
	
	\item Compute the matrix 
	\[
		\mQ = \int_{0}^1\int_0^1 \mH_{s,t} \ \d{s} \ \d{t}.
	\]
	(This is done simply by integrating each entry individually.)
	
	\item Describe how to set up and solve the optimization program
	\[
		\minimize_{\vf\in\setP_2}\sum_{m=1}^M(y_m - f(s_m,t_m))^2 + \delta \int_0^1\int_0^1\|\nabla f(s,t)\|_2^2\,\d{s}\d{t}.
	\]
	What is the regularizer above penalizing?  What kinds of solutions do we expect for large $\delta$?
	
	\item Apply your answer to part (d) to the data set from Problem 1.  Play around with the value of $\delta$, and produce estimates for three different $\delta$ that are interesting.  Discuss why you think those values are indeed ``interesting''.
\end{enumerate}


















%--------------------------------------------------------------------
\vspace{4mm} 
\item  Let $\mA$ be an $M\times N$ matrix with $\rank(\mA) < N$.  We have seen in this case that the least-squares problem
\begin{equation}
	\label{eq:ls}
	\minimize_{\vx\in\R^N} \|\vy-\mA\vx\|_2^2
\end{equation}
has an infinite number of solutions.  We have also seen, however, that the regularized least squares problem
\begin{equation}
	\label{eq:tikreg}
	\minimize_{\vx\in\R^N}\|\vy-\mA\vx\|_2^2 + \delta\|\vx\|_2^2
\end{equation}
has a unique solution for every $\delta > 0$.  In this problem, we will show that as $\delta\rightarrow 0$, the regularized solution goes to the minimum norm solution of
\begin{equation}
	\label{eq:minnorm}
	\minimize_{\vx\in\R^N}\|\vx\|_2^2\quad\text{subject to}\quad \mA^\T\mA\vx = \mA^\T\vy.
\end{equation}
\begin{enumerate}
	\item Start by showing that if $\vx_1\in\Row(\mA)$ and $\vx_2\in\Row(\mA)$ then $\mA^\T\mA\vx_1\not=\mA^\T\mA\vx_2$ unless $\vx_1=\vx_2$.
	\item Use part (a) to argue that the solution to \eqref{eq:minnorm} is always unique.
	\item In fact, something stronger than what we showed in part (a) is true. 	
	
	There exists a constant $C>0$ such that
	\[
		\|\mA^\T\mA(\vx_1-\vx_2)\|_2 ~\geq~ C\|\vx_1-\vx_2\|_2\quad\text{for all}\quad \vx_1,\vx_2\in\Row(\mA).
	\]
	(This follows very easily from work we do later in the course, so we will defer its proof for now.)  Use this fact to show that the solution of \eqref{eq:tikreg} goes to the solution of \eqref{eq:minnorm} as $\delta\rightarrow 0$.  In particular, if $\vx^\star$ is the (always unique) minimizer of \eqref{eq:minnorm}, and $\hat\vx_n$ is the (always unique) minimizer of \eqref{eq:tikreg} with\footnote{There is nothing special about taking $\delta=1/n$ ... your argument should work for any sequence of $\delta$s that goes to zero.} $\delta = 1/n$, show that
\[
	\lim_{n\rightarrow\infty}\hat\vx_n = \vx^\star,
\]
i.e.\ $\lim_{n\rightarrow\infty}\|\vx^\star-\hat\vx_n\|_2 = 0$.
\end{enumerate}
	





\end{enumerate}
\end{document}
